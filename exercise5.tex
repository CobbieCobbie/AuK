\documentclass[a4paper,12pt,headsepline]{scrartcl}

%\part{title}
\usepackage[utf8]{inputenc}
\usepackage{graphicx}
\usepackage{caption,subcaption}
\usepackage[british]{babel}
\usepackage[T1]{fontenc}
\usepackage{geometry}
\usepackage{proof}
\geometry{left=3.5cm, right=2cm, top=2.5cm, bottom=2cm}
\usepackage{hyperref}
%\usepackage[hyphens,obeyspaces,spaces]{url}
\usepackage{fancybox}
\usepackage{amssymb,amsmath,amsthm}
\usepackage{gensymb}
\usepackage[linesnumbered,ruled,vlined,norelsize]{algorithm2e}
%\usepackage[bookmarksnumbered,pdftitle={\titleDocument},hyperfootnotes=false]{hyperref} 
\usepackage{color}
\usepackage{float}
\usepackage{enumerate}
\usepackage{tikz}
\usetikzlibrary{positioning}
\usetikzlibrary{patterns}
\usepackage{pgfplots}
\pgfplotsset{compat=1.12}
\usepgfplotslibrary{fillbetween}

%test
%\usepackage[backend=bibtex]{biblatex}
%\usepackage{filecontents}

%\addbibresource{ref.bib}

\restylefloat{figure}

% Makros
%\newenvironment{sketch}{\begin{proof}[Proof (Sketch)]}{\end{proof}}
%\newtheorem{theorem}{Theorem}
%\newtheorem{assumption}{Assumption}
\newtheorem{lemma}{Lemma}
%\newtheorem{remark}{Remark}
%\newtheorem{definition}{Definition}
%\newtheorem{corollary}{Corollary}
\newcommand{\comment}[1]
{
  \begin{quotation}
    \textcolor{blue}{\underline{Edit:} #1}
  \end{quotation}
}
\newtheorem{aufgabe}{Exercise}
\newcommand{\Ex}[2]
{
	\setcounter{section}{#2}
	\section*{Übungsblatt #2 zu #1}
}
\newcommand{\TODO}[1]
{
  \begin{quotation}
    \textcolor{red}{\underline{TODO:} #1}
  \end{quotation}
}
% Zeichen 
\newcommand{\OO}{\ensuremath{\mathcal{O}}}
\newcommand{\ec}{\texttt{ec}}
\newcommand{\NP}{\call{NP}}
\newcommand{\call}[1]{\ensuremath{\mathcal{#1}}}

% neue Kopfzeilen mit fancypaket
\usepackage{fancyhdr} %Paket laden
\pagestyle{fancy} %eigener Seitenstil
\fancyhf{} %alle Kopf- und Fußzeilenfelder bereinigen
\fancyhead[L]{Benjamin \c Coban \\ Christoph Jabs}
\fancyhead[C]{Algorithmen und Komplexität \\ Blatt 5}
\fancyhead[R]{3526251 \\ 5567177}
\setlength{\headheight}{39pt}
\renewcommand{\headrulewidth}{0.4pt} %obere Trennlinie
%\fancyfoot[C]{\thepage} %Seitennummer
%\renewcommand{\footrulewidth}{0.4pt} %untere Trennlinie

\frenchspacing
\makeindex

% Pseudocode für Java
\usepackage{listings}
\lstset{numbers=left, numberstyle=\tiny, numbersep=5pt, keywordstyle=\color{black}\bfseries, stringstyle=\ttfamily,showstringspaces=false,basicstyle=\footnotesize,captionpos=b}
\lstset{language=java}

% Disable single lines at the start of a paragraph (Schusterjungen)
\clubpenalty = 10000
% Disable single lines at the end of a paragraph (Hurenkinder)
\widowpenalty = 10000
\displaywidowpenalty = 10000

\begin{document}

\begin{aufgabe}
\end{aufgabe}

In the graphical illustrations, these colours mark the following:
\begin{itemize}
  \item \textcolor{blue}{Blue area}: infeasible due to constraint 1
  \item \textcolor{red}{Red area}: infeasible due to constraint 2
  \item \textcolor{orange}{Orange area}: infeasible due to constraint 3
  \item \textcolor{green}{Green line}: line of solutions with value 0. The arrow marks the direction in which the value increases.
\end{itemize}

\begin{enumerate}[a)]
  \item\mbox{}\\
    \def\xm{6}
    \def\ym{6}
    \begin{center}
      \begin{tikzpicture}
        \begin{axis}[
          xmin=-1, xmax=\xm,
          ymin=-1, ymax=\ym,
          axis lines=middle,
          axis line style={->},
          xlabel={$x_1$},
          ylabel={$x_2$}
          ]  
          \path[name path=top] (axis cs:-1,\ym) -- (axis cs:\xm,\ym);
          \path[name path=bottom] (axis cs:-1,-1) -- (axis cs:\xm,-1);
          \path[name path=left] (axis cs:-1,-1) -- (axis cs:-1,\ym);
          \path[name path=right] (axis cs:\xm,-1) -- (axis cs:\xm,\ym);

          \addplot[blue, samples=100, domain=-1:\xm, name path=A] {-4*x+8}; 
          \addplot[pattern=north east lines, pattern color=blue] fill between[of=A and top];
          \addplot[red, samples=100, domain=-1:\xm, name path=B] {0.5*x+2}; 
          \addplot[pattern=north east lines, pattern color=red] fill between[of=B and bottom];
          \draw[orange, name path=C] (axis cs:1,-1) -- (axis cs:1,\ym);
          \addplot[pattern=north east lines, pattern color=orange] fill between[of=C and left];

          \addplot[green, domain=-1:\xm] {0.5*x};
          \draw[green, thick, shorten <=3pt, ->] (axis cs:0,0) -- (axis cs:-1/2,1);
        \end{axis}
      \end{tikzpicture}
    \end{center}

    The problem has a finite set of feasible solutions.
    We find the optimal solution by moving the green line in the direction of the arrow until it touches only one edge of the feasible set.
    In this case we therefore have only one optimal solution.

  \item\mbox{}\\
    \begin{center}
      \begin{tikzpicture}
        \begin{axis}[
          xmin=-1, xmax=\xm,
          ymin=-1, ymax=\ym,
          axis lines=middle,
          axis line style={->},
          xlabel={$x_1$},
          ylabel={$x_2$}
          ]  
          \path[name path=top] (axis cs:-1,\ym) -- (axis cs:\xm,\ym);
          \path[name path=bottom] (axis cs:-1,-1) -- (axis cs:\xm,-1);
          \path[name path=left] (axis cs:-1,-1) -- (axis cs:-1,\ym);
          \path[name path=right] (axis cs:\xm,-1) -- (axis cs:\xm,\ym);

          \addplot[blue, samples=100, domain=-1:\xm, name path=A] {-4*x+8}; 
          \addplot[pattern=north east lines, pattern color=blue] fill between[of=A and bottom];
          \addplot[red, samples=100, domain=-1:\xm, name path=B] {0.5*x+2}; 
          \addplot[pattern=north east lines, pattern color=red] fill between[of=B and top];
          \draw[orange, name path=C] (axis cs:1,-1) -- (axis cs:1,\ym);
          \addplot[pattern=north east lines, pattern color=orange] fill between[of=C and right];

          \addplot[green, domain=-1:\xm] {0.5*x};
          \draw[green, thick, shorten <=3pt, ->] (axis cs:0,0) -- (axis cs:-1/2,1);
        \end{axis}
      \end{tikzpicture}
    \end{center}

    There are no feasible points left over by the constraints, therefore the problem is infeasible.

  \item\mbox{}\\
    \begin{center}
      \begin{tikzpicture}
        \begin{axis}[
          xmin=-1, xmax=\xm,
          ymin=-1, ymax=\ym,
          axis lines=middle,
          axis line style={->},
          xlabel={$x_1$},
          ylabel={$x_2$}
          ]  
          \path[name path=top] (axis cs:-1,\ym) -- (axis cs:\xm,\ym);
          \path[name path=bottom] (axis cs:-1,-1) -- (axis cs:\xm,-1);
          \path[name path=left] (axis cs:-1,-1) -- (axis cs:-1,\ym);
          \path[name path=right] (axis cs:\xm,-1) -- (axis cs:\xm,\ym);

          \addplot[blue, samples=100, domain=-1:\xm, name path=A] {-4*x+8}; 
          \addplot[pattern=north east lines, pattern color=blue] fill between[of=A and bottom];
          \addplot[red, samples=100, domain=-1:\xm, name path=B] {0.5*x+2}; 
          \addplot[pattern=north east lines, pattern color=red] fill between[of=B and bottom];
          \draw[orange, name path=C] (axis cs:1,-1) -- (axis cs:1,\ym);
          \addplot[pattern=north east lines, pattern color=orange] fill between[of=C and left];

          \addplot[green, domain=-1:\xm] {0.5*x};
          \draw[green, thick, shorten <=3pt, ->] (axis cs:0,0) -- (axis cs:1/2,-1);
        \end{axis}
      \end{tikzpicture}
    \end{center}

    To optimize the problem, we start with a line parallel to the green one, but intersecting the feasible region.
    We then move this line in the direction of the green arrow until it is no longer intersecting the feasible region.
    Since in this case, the second constraint is parallel to the objective function, this will be when the green line is equal to the red line.
    In this case we have infinitely many optimal solutions.

  \item\mbox{}\\
    \begin{center}
      \begin{tikzpicture}
        \begin{axis}[
          xmin=-1, xmax=\xm,
          ymin=-1, ymax=\ym,
          axis lines=middle,
          axis line style={->},
          xlabel={$x_1$},
          ylabel={$x_2$}
          ]  
          \path[name path=top] (axis cs:-1,\ym) -- (axis cs:\xm,\ym);
          \path[name path=bottom] (axis cs:-1,-1) -- (axis cs:\xm,-1);
          \path[name path=left] (axis cs:-1,-1) -- (axis cs:-1,\ym);
          \path[name path=right] (axis cs:\xm,-1) -- (axis cs:\xm,\ym);

          \addplot[blue, samples=100, domain=-1:\xm, name path=A] {-4*x+8}; 
          \addplot[pattern=north east lines, pattern color=blue] fill between[of=A and bottom];
          \addplot[red, samples=100, domain=-1:\xm, name path=B] {0.5*x+2}; 
          \addplot[pattern=north east lines, pattern color=red] fill between[of=B and bottom];
          \draw[orange, name path=C] (axis cs:1,-1) -- (axis cs:1,\ym);
          \addplot[pattern=north east lines, pattern color=orange] fill between[of=C and right];

          \addplot[green, domain=-1:\xm] {0.5*x};
          \draw[green, thick, shorten <=3pt, ->] (axis cs:0,0) -- (axis cs:-1/2,1);
        \end{axis}
      \end{tikzpicture}
    \end{center}

    In this case we can move the green line infinitely far, therefore the problem is unbounded.
\end{enumerate}

\newpage

\begin{aufgabe}
\end{aufgabe}

\begin{enumerate}[a)]
  \item\label{it:feasible} Since the initial basic solution is not feasible, we define the auxiliary linear problem:
    \begin{equation*}
      \begin{aligned}
        \text{maximize}   & &      & &   & &      & & - & & x_0 &     &    \\
        \text{subject to} & & x_1  & & - & & x_2  & & - & & x_0 & \le & -3 \\
                          & & x_1  & & + & & x_2  & & - & & x_0 & \le & 7  \\
                          & & x_1, & &   & & x_2, & &   & & x_0 & \ge & 0  \\
      \end{aligned}
    \end{equation*}

    Now we can write the auxiliary problem in slackform and start solving it:

    \begin{center}
      \begin{tabular}{ccc}
        Dictionary & Entering Variable & Leaving Variable \\
        \hline
        \hline
      {$\begin{aligned}
        z &= -x_0 \\
        x_3 &= -3 -x_1 +x_2 +x_0 \\
        x_4 &= 7 -x_1 -x_2 +x_0
      \end{aligned}$} & $x_0$ & $x_3$ \\
      \hline
      {$\begin{aligned}
        z &= -3 -x_1 +x_2 -x_3 \\
        x_0 &= 3 +x_1 -x_2 +x_3 \\
        x_4 &= 10 -2x_2 +x_3
      \end{aligned}$} & $x_2$ & $x_0$ \\
      \hline
      {$\begin{aligned}
        z &= -x_0 \\
        x_2 &= 3 +x_1 +x_3 -x_0 \\
        x_4 &= 4 -2x_1 -x_3 +2x_0 \\
      \end{aligned}$} & 
      \end{tabular}
    \end{center}

    This dictionary is now optimal with optimal value 0 and solution $x_0=0$, $x_1=0$, $x_2=3$, $x_3=0$, $x_4=4$.
    With using the equalities from the constraints of the auxiliary problem, we get the following feasible dictionary for the original linear program:

    \begin{align*}
      z &= 2x_1 +3 +x_1 +x_3 = 3 +3x_1 +x_3 \\
      x_2 &= 3 +x_1 +x_3 \\
      x_4 &= 4 -2x_1 -x_3
    \end{align*}

  \item We now solve the linear program starting from the feasible dictionary computed in~\ref{it:feasible}.
    \begin{center}
      \begin{tabular}{ccc}
        Dictionary & Entering Variable & Leaving Variable \\
        \hline
        \hline
        {$\begin{aligned}
          z &= 3 +3x_1 +x_3 \\
          x_2 &= 3 +x_1 +x_3 \\
          x_4 &= 4 -2x_1 -x_3
        \end{aligned}$} & $x_1$ & $x_4$ \\
        \hline
        {$\begin{aligned}
          z &= 9 -2x_3 -1.5x_4 \\
          x_2 &= 5 -0.5x_4 \\
          x_1 &= 2 -x_3 -0.5x_4
        \end{aligned}$} & 
    \end{tabular}
  \end{center}

  We now have the final dictionary and can determine the optimal solution to be of value 9 and stem from the value $x_1=2$ and $x_2=5$.
\end{enumerate}

\newpage

\begin{aufgabe}
\end{aufgabe}

When looking at the constraints of the given linear program, we can insert the fact that $b=\vec 0$ and get $Ax\le \vec 0$.
Since $x\ge 0$, we can conclude from this that the linear program is never infeasible, since independently of $A$, if $x=\vec 0$, the constraint is always fulfilled.
From this we know that the problem either has an optimal solution or is unbounded.

For gaining information about those two cases, we look at the dual of the problem.
\begin{align*}
  \text{minimize}   & & y^T & & b & &     & &     \\
  \text{subject to} & & y^T & & A & & \ge & & c^T \\
                    & &     & & y & & \ge & & 0
\end{align*}
Since we know that the primal cannot be infeasible, we know that the dual cannot be unbounded.
When looking at the constraint $y^TA\ge c^T$, since we do not know anything about $A$ and $c$, we cannot say if it is infeasible or it has an optimal solution.
Therefore there are two possibilities:
\begin{enumerate}
  \item The dual is infeasible and the primal therefore unbounded
  \item The dual has an optimal solution.
    From the objective function of the dual $y^Tb=y^T\vec 0=0$, we know that this optimal solution has the value 0.
    The duality theorem tells us that the value of the optimal solution to the primal problem is equal to the optimal solution of the dual problem.
    Therefore we know that $c^Tx^*=0$ and since $c$ is any arbitrary vector, this solution does only hold for $x=\vec 0$.
\end{enumerate}

\newpage

\begin{aufgabe}
\end{aufgabe}

\begin{enumerate}[a)]
  \item Consider the following dictionary:
    \begin{align*}
      z &= x_1 +2x_3 \\
      x_2 &= 1 -x_1 -x_3
    \end{align*}
    When applying the simplex method with Bland's rule, $x_1$ will become basic in the next step and the dictionary will change to the following:
    \begin{align*}
      z &= 1 -x_2 +x_3 \\
      x_1 &= 1 -x_2 -x_3
    \end{align*}
    Since $x_3$ does now have a positive coefficient in the objective function, it will become basic in the next step and $x_1$ will therefore become non-basic again.
    \begin{align*}
      z &= -x_1 -x_2 \\
      x_3 &= 1 -x_1 -x_2
    \end{align*}

  \item For the leaving variable to become basic again in the next iteration, it would have to be the variable with the lowest index and a positive coefficient in the objective function.
    Consider the following excerpt of a dictionary:
    \begin{align*}
      z &= \dots +cx_e +\dots \\
      x_l &= b +\dots -ax_l +\dots
    \end{align*}
    Here $x_e$ shall be the entering and $x_l$ the leaving variable.
    For $x_e$ to be entering, $c>0$ needs to be true and for $x_l$ to be leaving, $a,b>0$ needs to be true as well.

    If we now express $x_e$ in terms of $x_l$, we get the following:
    \[ x_3 = \frac ba +\dots -\frac 1a x_l +\dots \]
    Plugging that into the objective function yields:
    \[ z = \dots +c \left(\frac ba +\dots -\frac 1a x_l +\dots\right) \]
    Since $x_l$ was basic in the last step, is cannot have been present in the objective function.
    Therefore the final coefficient of $x_l$ in the objective function is $-\frac ca<0$.
    This is why $x_l$ cannot become basic in the next step.
\end{enumerate}

\newpage
\begin{aufgabe}
\end{aufgabe}

The following linear problem is to be solved with the Simplex algorithm, whereas the the Blands rule shall be applied. \textbf{The Blands Rule} states that in a degenerate case, the choice to pick lies in the smallest index.
\begin{align*}
    &\textbf{\text{maximize}} &10x_1 - 57x_2 - 9x_3 - 24x_4 & \\
    &\textbf{\text{subject to}} & \frac{1}{2}x_1 -  \frac{11}{2} x_2 -  \frac{5}{2}x_3 +  9 x_4 & \leq  0 \\
    & &\frac{1}{2}x_1 -\frac{3}{2}x_2-\frac{1}{2}x_3+x_4&\leq 0 \\
    & &x_1 & \leq 1
\end{align*}
Where $x_i \geq 0$. In the following, $s_i$ and $z$ are slack variables regarding the constraints and the maximizing function, respectively.\\
\begin{tabular}{c|c|c|c|c|c|c|c|c|}
     & $x_1$ & $x_2$ & $x_3$ & $x_4$ & $s_1$ & $s_2$ & $s_3$ & RHS  \\
\hline   $s_1$ & $ \frac{1}{2}$& $-\frac{11}{2} $ &$-\frac{5}{2} $ & 9 & 1&0 & 0 & 0\\
\hline   $s_2$ &$\frac{1}{2} $ &$ -\frac{3}{2} $ & $-\frac{1}{2} $& 1 & 0&1 &0 & 0\\
\hline   $s_3$ & 1 & 0 & 0 & 0 & 0&0 &1 & 1\\
\hline   \hline$z$ & -10 & 57 & 9 & 24 & 0 & 0  & 0 & 0\\
\hline 
\end{tabular}\\     
It is clear that the row of choice is the first one, due to $-10x_1$ in the slack equation for $z$. The min test results in a tie between $s_1$ and $s_2$. Blands Rule states that in this case the choice lies in the index, in this case in $s_1$. Therefore, the following table is derived by basic operations:\\
\begin{tabular}{c|c|c|c|c|c|c|c|c|}
     & $x_1$ & $x_2$ & $x_3$ & $x_4$ & $s_1$ & $s_2$ & $s_3$ & RHS  \\
\hline   $x_1$ & $ \frac{1}{2}$& $-\frac{11}{2} $ &$-\frac{5}{2} $ & 9 & 1&0 & 0 & 0\\
\hline $s_2$ & 0 & 4 & 4 & -8 &-1 &1 & 0 & 0\\
\hline $s_3$ & 0 & $11$ &5 &-18 &-2 & 0&1 & 1\\
\hline  \hline $z$ & 0 & -53 & -41 & -156 & 20 & 0 & 0 & 0\\
\hline 
\end{tabular}\\     
The minimal coeffizient in $z$ is the one belonging to $x_4$. This time $s_3$ wins the min test.\\
\begin{tabular}{c|c|c|c|c|c|c|c|c|}
     & $x_1$ & $x_2$ & $x_3$ & $x_4$ & $s_1$ & $s_2$ & $s_3$ & RHS  \\
\hline   $x_1$ & 1& 0 &0 & 0 & 0&0 & 1 & 1\\
\hline $s_2$ & 0 & 8 & -16 & 0 &1 & 9 & 4 & 4\\
\hline $x_4$ & 0 & $11$ &5 &-18 &-2 & 0&1 & 1\\
\hline \hline $z$ & 0 & 445 & 253 & 0 & -112 & 0 & 0 & 0\\
\hline 
\end{tabular}\\ 
The row of choice is the one with the -112 entry in $z$, $s_1$. The min test is won by $x_4$.\\
\begin{tabular}{c|c|c|c|c|c|c|c|c|}
     & $x_1$ & $x_2$ & $x_3$ & $x_4$ & $s_1$ & $s_2$ & $s_3$ & RHS  \\
\hline   $x_1$ & 1& 0 &0 & 0 & 0&0 & 1 & 1\\
\hline $s_2$ & 0 & 27 & -27 & -18 &0 & 18 & 9 & 9\\
\hline $s_1$ & 0 & $11$ &5 &-18 &-2 & 0 & 1 & 1\\
\hline \hline $z$ & 0 & 171 & 27 & -1008 & 0 & 0 & 56 & 56\\
\hline 
\end{tabular}\\
The row of choice belongs to $x_4$, the min test is won by $s_2$.\\
\begin{tabular}{c|c|c|c|c|c|c|c|c|}
     & $x_1$ & $x_2$ & $x_3$ & $x_4$ & $s_1$ & $s_2$ & $s_3$ & RHS  \\
\hline   $x_1$ & 1& 0 &0 & 0 & 0&0 & 1 & 1\\
\hline $x_4$ & 0 & 27 & -27 & -18 &0 & 18 & 9 & 9\\
\hline $s_1$ & 0 & 16 &-32 &0 & 2 & 18 & 8 & 8\\
\hline \hline $z$ & 0 & 1341 & -1539 & 0 & 0 & 1008 & 448 & 448\\
\hline 
\end{tabular}\\
The row of choice belongs to $x_3$, the min test is won by $x_4$:\\
\begin{tabular}{c|c|c|c|c|c|c|c|c|}
     & $x_1$ & $x_2$ & $x_3$ & $x_4$ & $s_1$ & $s_2$ & $s_3$ & RHS  \\
\hline   $x_1$ & 1& 0 &0 & 0 & 0&0 & 1 & 1\\
\hline $x_3$ & 0 & 27 & -27 & -18 &0 & 18 & 9 & 9\\
\hline $s_1$ & 0 & -432 &0 &576 & -54 & -90 & -72 & -72\\
\hline \hline $z$ & 0 & 198 & 0 & -1026 & 0 & 18 & 65 & 65\\
\hline 
\end{tabular}\\
The row of choice belongs to $x_4$, the min test winner is $x_3$:\\
\begin{tabular}{c|c|c|c|c|c|c|c|c|}
     & $x_1$ & $x_2$ & $x_3$ & $x_4$ & $s_1$ & $s_2$ & $s_3$ & RHS  \\
\hline   $x_1$ & 1& 0 &0 & 0 & 0&0 & 1 & 1\\
\hline $x_4$ & 0 & 27 & -27 & -18 &0 & 18 & 9 & 9\\
\hline $s_1$ & 0 & -432 &0 &576 & -54 & -90 & -72 & -72\\
\hline \hline $z$ & 0 & 198 & 0 & -1026 & 0 & 18 & 65 & 65\\
\hline 
\end{tabular}\\

\newpage
\begin{aufgabe}
\end{aufgabe}



\newpage
\begin{aufgabe}
\end{aufgabe}

\end{document}

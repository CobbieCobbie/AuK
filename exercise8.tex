\documentclass[a4paper,12pt,headsepline]{scrartcl}

%\part{title}
\usepackage[utf8]{inputenc}
\usepackage{graphicx}
\usepackage{caption,subcaption}
\usepackage[british]{babel}
\usepackage[T1]{fontenc}
\usepackage{geometry}
\usepackage{proof}
\geometry{left=3.5cm, right=2cm, top=2.5cm, bottom=2cm}
\usepackage{hyperref}
%\usepackage[hyphens,obeyspaces,spaces]{url}
\usepackage{fancybox}
\usepackage{amssymb,amsmath,amsthm}
\usepackage{gensymb}
\usepackage[linesnumbered,ruled,vlined,norelsize]{algorithm2e}
%\usepackage[bookmarksnumbered,pdftitle={\titleDocument},hyperfootnotes=false]{hyperref} 
\usepackage{color}
\usepackage{float}
\usepackage{enumerate}
\usepackage{marvosym}
\usepackage{tikz}
\usetikzlibrary{positioning}
\usetikzlibrary{patterns}
\usepackage{pgfplots}
\pgfplotsset{compat=1.12}
\usepgfplotslibrary{fillbetween}
%%%

% Always forgetting the figure parameters for precise graphical inclusions -.-

%\begin{figure}[H]
	%\centering
	%\begin{subfigure}{0.4\textwidth}
		%\centering
		%\includegraphics[width=0.34\linewidth,page=6]{includegraphics/L-%t-shape_candidates}
%		\caption{Empty $T$-face}\label{im:empty_T}	
%	\end{subfigure}
%%%
%test
%\usepackage[backend=bibtex]{biblatex}
%\usepackage{filecontents}

%\addbibresource{ref.bib}

\restylefloat{figure}

% Makros
%\newenvironment{sketch}{\begin{proof}[Proof (Sketch)]}{\end{proof}}
%\newtheorem{theorem}{Theorem}
%\newtheorem{assumption}{Assumption}
\newtheorem{lemma}{Lemma}
%\newtheorem{remark}{Remark}
%\newtheorem{definition}{Definition}
%\newtheorem{corollary}{Corollary}
\newcommand{\comment}[1]
{
  \begin{quotation}
    \textcolor{blue}{\underline{Edit:} #1}
  \end{quotation}
}
\newtheorem{aufgabe}{Exercise}
\newcommand{\Ex}[2]
{
	\setcounter{section}{#2}
	\section*{Übungsblatt #2 zu #1}
}
\newcommand{\TODO}[1]
{
  \begin{quotation}
    \textcolor{red}{\underline{TODO:} #1}
  \end{quotation}
}
% Zeichen 
\newcommand{\OO}{\ensuremath{\mathcal{O}}}
\newcommand{\ec}{\texttt{ec}}
\newcommand{\NP}{\call{NP}}
\newcommand{\call}[1]{\ensuremath{\mathcal{#1}}}

% neue Kopfzeilen mit fancypaket
\usepackage{fancyhdr} %Paket laden
\pagestyle{fancy} %eigener Seitenstil
\fancyhf{} %alle Kopf- und Fußzeilenfelder bereinigen
\fancyhead[L]{Benjamin \c Coban \\ Christoph Jabs}
\fancyhead[C]{Algorithmen und Komplexität \\ Blatt 8}
\fancyhead[R]{3526251 \\ 5567177}
\setlength{\headheight}{39pt}
\renewcommand{\headrulewidth}{0.4pt} %obere Trennlinie
%\fancyfoot[C]{\thepage} %Seitennummer
%\renewcommand{\footrulewidth}{0.4pt} %untere Trennlinie

\frenchspacing
\makeindex

% Pseudocode für Java
\usepackage{listings}
\lstset{numbers=left, numberstyle=\tiny, numbersep=5pt, keywordstyle=\color{black}\bfseries, stringstyle=\ttfamily,showstringspaces=false,basicstyle=\footnotesize,captionpos=b}
\lstset{language=java}

% Disable single lines at the start of a paragraph (Schusterjungen)
\clubpenalty = 10000
% Disable single lines at the end of a paragraph (Hurenkinder)

\widowpenalty = 10000
\displaywidowpenalty = 10000
\begin{document}
\begin{aufgabe}ILP
\end{aufgabe}
\begin{enumerate}
	\item ILP with enumerated vertices:
	\begin{align*}
		\textbf{\text{maximize}}& \sum x_i\\
		\textbf{\text{s.t.}}& x_i + x_j \leq 1 \forall \{i,j\}\in E\\
		& x_i = 0 \text{ or } 1 \forall i \in V
	\end{align*}
\end{enumerate}
\newpage
\begin{aufgabe}Planar graph properties
\end{aufgabe}
Following Eulers Theorem:
\begin{align*}
n + f = m + 2
\end{align*}
A maximal planar graph has exactly $3n-6$ edges, meaning:
\begin{align*}
n + f = 3n-6 + 2   \Rightarrow f = 2n-4
\end{align*}
\begin{enumerate}[a)]
	\item It is to proof that for a planar embedded graph $G$ with $n$ vertices and $m$ edges without any triangles the following holds:
	\begin{align*}
		m \leq 2n-4
	\end{align*}
	\begin{proof}
		Consider a maximal planar graph $G'$ with $n$ vertices. Then, it holds that $G'$ has exactly $m' = 3n-6$ edges and $f' = 2n-4$ faces. Since $G'$ is maximal planar, every face is a triangle. A planar graph $G$ without any triangles can be constructed by removing an edge between two adjacent faces. Then, two triangles become one quadrangle. Since $G'$ has exactly $2n-4$ edges, there are $n-2$ edge removals to obtain a planar graph without any triangles. Therefore $G$ can hold at most $m \leq m' - (v-2) = 2n -4$ edges.
	\end{proof}
	\item It is to examine what an upper bound of edge counts might be when a planar embedded graph $G'$ ain't got any cycles of size three or four. The following holds:
	\begin{align*}
		m \leq \frac{5}{3}n - \frac{10}{3}
	\end{align*}
	\begin{proof}
		Again, a maximal planar graph $G'$ with $n$ vertices holds exactly $m' = 3n-6$ edges and $f' = 2n-4$ faces. From $G'$, the removal of two neighbouring edges of three adjacent triangles will result in a residual graph $G$ with cycles of size at least 5. So, in count, for every three triangles two edges will be removed. Therefore it holds, that $m \leq m' - \frac{2\cdot (2n-4)}{3} = 3n-6 - (\frac{4}{3}n - \frac{8}{3}) = \frac{5}{3}n - \frac{10}{3}$.
	\end{proof}
\end{enumerate}
\newpage
\begin{aufgabe}Planarity
\end{aufgabe}
\begin{enumerate}[b)]
	\item
	\item 
	\item 
	\item
	\item
\end{enumerate}
\newpage
\begin{aufgabe}1-planar graphs
\end{aufgabe}
\begin{enumerate}
	\item A maximal planar graph $G$ has exactly $3n-6$ edges. Every further edge insertion with respect to the 1-planarity constraint will result exactly one crossing. 
	\item 
	\begin{proof}
		A triangulated planar graph is maximal planar. Suppose not. Then, there would be at least one edge $e (v_1,v_3)$ which could be inserted without violating the planarity. Then, $v_1,v_3$ are at least part of a circle of size $4$. Then the graph is not triangulated. Following Eulers Theorem:
		\begin{align*}
		n + f = m + 2
		\end{align*}
		A maximal planar graph has exactly $3n-6$ edges, meaning:
		\begin{align*}
		n + f = 3n-6 + 2   \Rightarrow f = 2n-4
		\end{align*}
	\end{proof}
\end{enumerate}
\end{document}
